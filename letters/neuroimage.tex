\documentclass[12pt]{letter}
\usepackage{hyperref}

% page counting, header/footer
\usepackage{fancyhdr}
\usepackage{lastpage}
\usepackage{graphicx}

\pagestyle{fancy}
\fancyhead[RO,RE]{\includegraphics[width=3cm]{letters/camh_logo.png}}
%\renewcommand{\headheight}{24pt}
%\renewcommand{\footrulewidth}{0.4pt}

\signature{Jon Pipitone \\
Kimel Family Translational Imaging-Genetics Laboratory \\
Research Imaging Centre \\
Centre for Addiction and Mental Health \\
Toronto, Ontario}
\address{}
\begin{document}
 
\begin{letter}{Editorial Board, Neuroimage}
\opening{Dear Sir or Madam:}
\thispagestyle{fancy}

Please find enclosed our manuscript entitled, {\em Bootstrapping Multi-atlas
Segmentation Using Multiple Automatically Generated Templates for the
Segmentation of the Whole Hippocampus and Subfields} by Jon Pipitone and
colleagues.

In this manuscript we describe a novel automated MRI hippocampal segmentation
algorithm optimised to perform well with only a small number of manually
segmented training images. We believe this is an important contribution because
the expertise and effort needed to perform manual segmentation can be
prohibitive for many clinicians and researchers, and yet existing automated
methods generally require between 30 to 80 training segmentations. This
situation makes it infeasible to use these methods for segmentations based on
histological-based digital segmentations (because of their rarity),
high-resolution digital atlases (because of the time needed to segment these
images), or when exploring the effect of variations on a segmentation protocol.

It is for these reasons that we developed the automated segmentation algorithm,
mischievously named MAGeT-Brain, which takes advantage of the neuroanatomical
variability that exists in the target population being studied to bootstrap a
large template library from a small set of manually segmented images. In this
manuscript we rigorously validate this approach on multiple disease
populations, and compare our segmentations with existing popular methods (e.g.
FSL and FreeSurfer).  Our results demonstrate consistent accuracy in the
identification of the whole hippocampus when compared to "gold-standard" manual
segmentations.  Finally, we have made our algorithm available publically online
for use by other groups, and are pursuing the contribution of our segmentations
to the Alzheimer's Disease Neuroimaging Initiative image database.

We believe that the technique we have developed and our findings are a
significant contribution to the neuroimaging community, specifically for those
researchers interested in large scale studies of the hippocampus in the context
of normal brain function and different forms of brain dysfunction.

We hope you find the enclosed manuscript meets the high standards of
NeuroImage.

 
\closing{Sincerely,}
 
\encl{Manuscript}
 
\end{letter}
\end{document}
